% !TEX TS-program = lualatex
% !TEX encoding = UTF-8

\documentclass[Session2024.tex]{subfiles}

\ifcsname preamble@file\endcsname
  \setcounter{page}{\getpagerefnumber{M-01_dimanche}}
\fi

\begin{document}

\bigtitle{16\textsuperscript{e} dimanche \emph{per annum}, aux Vêpres}{dimanche 21 juillet}{Vêpres}

\gscore{or_dia_festivus}

\gscore{hy_o_lux_beata}
\translation{TODO}

\gscore{an_dixit_dominus}
\translation{TODO}
\psalm{109}{7}

\gscore{an_fidelia_omnia}
\translation{TODO}
\psalm{110}{4}

\gscore{an_in_mandatis_eius}
\translation{TODO}
\psalm{111}{4}

\gscore{an_sit_nomen_domini}
\translation{TODO}
\psalm{112}{7}

\capitulus{Hebr 12, 22-24}
{Accessístis ad Sion montem et civitátem Dei vivéntis, Ierúsalem
cæléstem, et multa mília angelórum, † frequéntiam et ecclésiam
primogenitórum qui conscrípti sunt in cælis, et iúdicem Deum ómnium et
spíritus iustórum qui consummáti sunt, * et testaménti novi mediatórem
Iesum et sánguinem aspersiónis mélius loquéntem quam Abel.}
{Vous êtes venus vers la montagne de Sion et vers la cité du Dieu vivant,
la Jérusalem céleste, vers des milliers d’anges en fête et vers l’assemblée
des premiers-nés dont les noms sont inscrits dans les cieux. Vous êtes
venus vers Dieu, le juge de tous les hommes, et vers les âmes des justes
arrivés à la perfection. Vous êtes venus vers Jésus, le médiateur d’une
Alliance nouvelle, et vers son sang répandu sur les hommes, son sang qui
parle plus fort que celui d’Abel.}

\gscore{rb_magnus_dominus}
\translation{TODO}

\gscore{an_videns_iesus_multam_turbam}
\translation{TODO}
\incipit{magn}{1fsol}
\psalm{magn}{1sol}

\gscore{or_kyrie_festivus}
\gscore{or_pater_festivus}

\label{TO16}
\oratio{
Propitiáre, Dómine, fámulis tuis, et clémenter grátiæ tuæ super eos
dona multíplica, † ut, spe, fide et caritáte fervéntes, * semper in mandátis
tuis vígili custódia persevérent. Per Dóminum.
}{
Sois favorable à tes fidèles, Seigneur, et multiplie les dons de ta grâce:
entretiens en eux la foi, l’espérance et la charité, pour qu’ils soient attentifs à
garder tes commandements. Par Jésus-Christ.
}

\gscore{or_benedictio_festivus}

\gscore{or_benedicamus_dominicis}

\end{document}