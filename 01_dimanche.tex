% !TEX TS-program = lualatex
% !TEX encoding = UTF-8

\documentclass[Session2024.tex]{subfiles}

\ifcsname preamble@file\endcsname
  \setcounter{page}{\getpagerefnumber{M-01_dimanche}}
\fi

\begin{document}

\bigtitle{16\textsuperscript{e} dimanche \emph{per annum}, aux Vêpres}{dimanche 21 juillet}{Vêpres}

\smallscore{or_dia_festivus}

\gscore{hy_o_lux_beata}
\translation{\colored{L}umière, heureuse Trinité, qui es souveraine Unité, quand l’astre de feu se retire, répands en nos cœurs ta clarté.\\
\colored{À} toi nos hymnes du matin, à toi nos cantiques du soir, à toi, pour les siècles des siècles, la prière de notre gloire.\\
\colored{N}ous prions le Christ et le Père, et l’Esprit du Père et du Christ : ô Trinité une et puissante, daigne écouter ceux qui te prient.}

\smalltitle{Psaume 109}
\gscore{an_dixit_dominus}
\translation{\aa Oracle du Seigneur à mon seigneur : Siège à ma droite.}
\psalm{109}{7}

\smalltitle{Psaume 110}
\gscore{an_fidelia_omnia}
\translation{\aa Sécurité, toutes ses lois, établies pour toujours et à jamais.}
\psalm{110}{4}

\smalltitle{Psaume 111}
\gscore{an_in_mandatis_eius}
\translation{\aa La volonté du Seigneur, il l'aime entièrement.}
\psalm{111}{4}

\smalltitle{Psaume 112}
\gscore{an_sit_nomen_domini}
\translation{\aa Béni soit le Nom du Seigneur, pour les siècles des siècles.}
\psalm{112}{7}

\capitulum{Hebr 12, 22-24}
{Accessístis ad Sion montem et civitátem Dei vivéntis, Ierúsalem
cæléstem, et multa mília angelórum, † frequéntiam et ecclésiam
primogenitórum qui conscrípti sunt in cælis, et iúdicem Deum ómnium et
spíritus iustórum qui consummáti sunt, * et testaménti novi mediatórem
Iesum et sánguinem aspersiónis mélius loquéntem quam Abel.}
{Vous êtes venus vers la montagne de Sion et vers la cité du Dieu vivant,
la Jérusalem céleste, vers des milliers d’anges en fête et vers l’assemblée
des premiers-nés dont les noms sont inscrits dans les cieux. Vous êtes
venus vers Dieu, le juge de tous les hommes, et vers les âmes des justes
arrivés à la perfection. Vous êtes venus vers Jésus, le médiateur d’une
Alliance nouvelle, et vers son sang répandu sur les hommes, son sang qui
parle plus fort que celui d’Abel.}

\gscore{rb_magnus_dominus}
\translation{TODO}

\smalltitle{Magnificat}
\gscore{an_videns_iesus_multam_turbam}
\translation{\aa Jésus vit une grande foule. Il fut saisi de compassion envers eux, parce qu’ils étaient comme des brebis sans berger.}
\incipit{magn}{1fsol}
\psalm{magn}{1sol}

\smalltitle{Intercessions}
\smallscore{or_kyrie_festivus}
\smallscore{or_pater_festivus}

\label{TO16}
\oratio{
Propitiáre, Dómine, fámulis tuis, et clémenter grátiæ tuæ super eos
dona multíplica, † ut, spe, fide et caritáte fervéntes, * semper in mandátis
tuis vígili custódia persevérent. Per Dóminum.
}{
Sois favorable à tes fidèles, Seigneur, et multiplie les dons de ta grâce:
entretiens en eux la foi, l’espérance et la charité, pour qu’ils soient attentifs à
garder tes commandements. Par Jésus-Christ.
}

\blessing

\gscore{or_benedicamus_dominicis}

\end{document}