% !TEX TS-program = lualatex
% !TEX encoding = UTF-8

\documentclass[Session2024.tex]{subfiles}

\ifcsname preamble@file\endcsname
  \setcounter{page}{\getpagerefnumber{M-03_mardi}}
\fi

\begin{document}

\bigtitle{Sainte Brigitte, à la Messe}{mardi 23 juillet}{Messe}

\gscore{in_cognovi_domine}
\translation{TODO}

\smalltitle{Psalmodie de Tierce}
\gscore{an_christus_nobis_dicit}
\translation{TODO}
\smalltitle{Psaume 119}
\psalm{119}{7}
\smalltitle{Psaume 120}
\psalm{120}{7}
\smalltitle{Psaume 121}
\psalm{121}{7}

\gscore{ky_k12}
\gscore{ky_g12}
\gscore{gr_diffusa_est_gratia}
\translation{TODO}
\gscore{al_specie_tua}
\translation{TODO}
\gscore{of_diffusa_est_gratia}
\translation{TODO}
\gscore{ky_s12}
\gscore{ky_a12}
\gscore{co_dilexisti_iustitiam}
\translation{TODO}

\bigtitle{Sainte Brigitte, à Sexte}{mardi 23 juillet}{Sexte}

\smallscore{or_dia_ferialis}

\gscore{hy_rector_potens_festivus}
\translation{TODO}

\gscore{an_haec_accipiet_benedictionem}
\translation{TODO}
\smalltitle{Psaume 122}
\psalm{122}{7}
\smalltitle{Psaume 123}
\psalm{123}{7}
\smalltitle{Psaume 124}
\psalm{124}{7}

\capitulum{1 Co 9, 26-27a}
{Ego sic curro non quasi in incértum, † sic pugno non quasi ærem
vérberans ; * sed castígo corpus meum et in servitútem rédigo.}
{Moi, si je cours, ce n’est pas sans fixer le but ; si je fais de la lutte, ce
n’est pas en frappant dans le vide. Mais je traite durement mon corps,
et j’en fais mon esclave.}

\versiculus{Invéni quem díligit ánima mea}{Ténui eum, nec dimíttam}{J’ai trouvé celui que mon coeur aime}{Je l’ai saisi, je ne le lâcherai pas}

\rubric{Oraison des Vêpres, p. \pageref{0723V}.}

\gscore{or_benedicamus_hm_festivus}

\bigtitle{Sainte Brigitte, à None}{mardi 23 juillet}{None}

\smallscore{or_dia_ferialis}

\gscore{hy_rerum_deus_festivus}
\translation{TODO}

\gscore{an_non_vos_me_elegistis}
\translation{TODO}
\smalltitle{Psaume 125}
\psalm{125}{4}
\smalltitle{Psaume 126}
\psalm{126}{4}
\smalltitle{Psaume 127}
\psalm{127}{4}

\capitulum{Ph 4, 8. 9b}
{Fratres, quæcúmque sunt vera, quæcúmque pudíca, quæcúmque
iusta, quæcúmque casta, † quæcúmque amabília, quæcúmque bonæ
famæ, si qua virtus et si qua laus, hæc cogitáte ; * et Deus pacis erit
vobíscum.}
{Enfin, mes frères, tout ce qui est vrai et noble, tout ce qui est juste et pur,
tout ce qui est digne d’être aimé et honoré, tout ce qui s’appelle vertu
et qui mérite des éloges, tout cela, prenez-le en compte. Et le Dieu de la
paix sera avec vous.}

\versiculus{Cantábo tibi, Dómine}{Psallam et intéllegam in via immaculáta}{À toi mes hymnes, Seigneur}{Je chanterai en suivant le chemin le plus parfait}

\rubric{Oraison des Vêpres, p. \pageref{0723V}.}

\gscore{or_benedicamus_hm_festivus}

\bigtitle{Sainte Brigitte, aux Vêpres}{mardi 23 juillet}{Vêpres}

\smallscore{or_dia_festivus}

\gscore{hy_fortem_virili_pectore}
\translation{TODO}

\smalltitle{Psaume 109}
\gscore{an_accinxit_fortitudine}
\translation{TODO}
\psalm{109}{8}

\smalltitle{Psaume 112}
\gscore{an_via_iustorum}
\translation{TODO}
\psalm{112}{8}

\smalltitle{Psaume 121}
\gscore{an_cognovit_eam}
\translation{TODO}
\psalm{121}{8}

\smalltitle{Psaume 126}
\gscore{an_lauda_qui_posuit}
\translation{TODO}
\psalm{126}{7}

\capitulum{Rm 8, 28-30}
{Scimus quóniam diligéntibus Deum ómnia cooperántur in bonum, *
his qui secúndum propósitum vocáti sunt. Nam, quos præscívit,
et prædestinávit confórmes fíeri imáginis Fílii eius, * ut sit ipse
primogénitus in multis frátribus ; / quos autem prædestinávit, hos et
vocávit ; † et quos vocávit, hos et iustificávit ; * quos autem iustificávit,
illos et glorificávit.}
{Nous le savons, quand les hommes aiment Dieu, lui-même fait tout
contribuer à leur bien, puisqu’ils sont appelés selon le dessein de son
amour. Ceux que, d’avance, il connaissait, il les a aussi destinés d’avance
à être configurés à l’image de son Fils, pour que ce Fils soit le premierné
d’une multitude de frères. Ceux qu’il avait destinés d’avance, il les a
aussi appelés ; ceux qu’il a appelés, il en a fait des justes ; et ceux qu’il
rendus justes, il leur a donné sa gloire.}

\gscore{rb_elegit_eam}
\translation{Dieu l'a choisie, il l'a prédestinée. Il l'a fait habiter dans sa demeure.}

\smalltitle{Magnificat}
\gscore{an_ego_dominus_sponsabo_te}
\translation{Moi, le Seigneur, je te fiancerai à moi pour toujours,
je te fiancerai à moi dans la justice et le droit, dans la fidélité et la tendresse ;
je te fiancerai à moi dans la loyauté.}
\incipit{magn}{7dsol}
\psalm{magn}{7sol}

\smalltitle{Intercessions}
\smallscore{or_kyrie_festivus}
\smallscore{or_pater_festivus}

\label{0723V}
\oratio{
Deus, qui beátam Birgíttam per várias vitae sémitas duxísti,
eámque sapiéntiam crucis in contemplatióne passiónis Fílii tui
mirabíliter docuísti, † concéde nobis, ut, digne in tua vocatióne
ambulántes, * te in ómnibus quæ´ rere valeámus. Per Dóminum.
}{
Seigneur Dieu, tu as conduit sainte Brigitte par divers chemins de vie,
et tu lui as enseigné de façon admirable la sagesse de la croix par la
contemplation de la Passion de ton Fils ; accorde à chacun de nous, quel
que soit son état de vie, de savoir te chercher en toute chose.
}

\blessing

\gscore{or_benedicamus_festis}

\end{document}
