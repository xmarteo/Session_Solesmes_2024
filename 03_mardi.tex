% !TEX TS-program = lualatex
% !TEX encoding = UTF-8

\documentclass[Session2024.tex]{subfiles}

\ifcsname preamble@file\endcsname
  \setcounter{page}{\getpagerefnumber{M-03_mardi}}
\fi

\begin{document}
%%% début en page de gauche
\addcontentsline{toc}{chapter}{Mardi 23 juillet, sainte Brigitte}
\bigtitle{Sainte Brigitte, à la Messe}{mardi 23 juillet}{Messe}

\gscore{in_cognovi_domine}
\translation{\aa J'ai reconnu, Seigneur, que tes jugements sont équitables, et que tu m'as humilié selon ta justice.
Transperce ma chair par ta crainte; je redoute tes jugements.\\
\vv Heureux ceux qui sont immaculés dans la voie, qui marchent dans la loi du Seigneur.}

\pagebreak
\smalltitle{Psalmodie de Tierce}
\gscore{an_christus_nobis_dicit}
\translation{\aa Le Christ nous dit dans l'Évangile: celui qui ne renonce pas à tout ce qu'il possède, ne peut être mon disciple.}
\smalltitle{Psaume 119}
\psalm{119}{7}
\smalltitle{Psaume 120}
\psalm{120}{7}
\smalltitle{Psaume 121}
\psalm{121}{7}

\gscore{ky_k12}
\gscore{ky_g12}
\gscore{gr_diffusa_est_gratia}
\translation{\rr La grâce est répandue sur tes lèvres; c'est pourquoi Dieu t'a bénie à jamais.
\vv Pour la vérité, la douceur et la justice; et ta droite te conduira merveilleusement.}
\gscore{al_specie_tua}
\translation{\vv Avec ta gloire et ta majesté, avance, marche victorieusement, et règne.}
\gscore{of_diffusa_est_gratia}
\translation{\rr La grâce est répandue sur tes lèvres; c'est pourquoi Dieu t'a bénie à jamais, dans les siècles.}
\newpage
\gscore{ky_s12}
\gscore{ky_a12}
\newpage
\gscore{co_dilexisti_iustitiam}
\translation{\aa Tu as aimé la justice, et haï l'iniquité; c'est pourquoi Dieu t'a ointe d'une huile d'allégresse d'une manière plus excellente que toutes tes compagnes.\\
\vv D'heureuses paroles jaillissent de mon coeur quand je dis mes poèmes pour le roi.\\
\vv Écoute, ma fille, regarde et tends l'oreille ; oublie ton peuple et la maison de ton père.}

\bigtitle{Sainte Brigitte, à Sexte}{mardi 23 juillet}{Sexte}

\smallscore{or_dia_ferialis}

\gscore{hy_rector_potens_festivus}
\translation{\colored{M}aître puissant, Dieu Vérité, tu règles la marche du temps, tu formes l’aube en sa clarté, au midi tu donnes ses flammes.\\
\colored{E}teins le feu des dissensions, calme la fièvre du péché, apporte à nos corps la santé, à nos cœurs, la paix véritable.\\
\colored{E}xauce-nous, Père très bon, et toi, le Fils égal au Père, avec l’Esprit Consolateur, régnant pour les siècles des siècles.}

\gscore{an_haec_accipiet_benedictionem}
\translation{\aa Celle-là recevra la bénédiction du Seigneur,
et sa miséricorde: car telle est la race
de ceux qui cherchent le Seigneur.}
\smalltitle{Psaume 122}
\psalm{122}{7}
\smalltitle{Psaume 123}
\psalm{123}{7}
\smalltitle{Psaume 124}
\psalm{124}{7}

\capitulum{1 Co 9, 26-27a}
{Ego sic curro non quasi in incértum, † sic pugno non quasi ærem
vérberans ; * sed castígo corpus meum et in servitútem rédigo.}
{Moi, si je cours, ce n’est pas sans fixer le but ; si je fais de la lutte, ce
n’est pas en frappant dans le vide. Mais je traite durement mon corps,
et j’en fais mon esclave.}

\versiculus{Invéni quem díligit ánima mea}{Ténui eum, nec dimíttam}{J’ai trouvé celui que mon coeur aime}{Je l’ai saisi, je ne le lâcherai pas}

\rubric{Oraison des Vêpres, p. \pageref{0723V}.}

\gscore{or_benedicamus_hm_festivus}

\bigtitle{Sainte Brigitte, à None}{mardi 23 juillet}{None}

\smallscore{or_dia_ferialis}

\gscore{hy_rerum_deus_festivus}
\translation{\colored{D}ieu fort, soutien de l’univers, qui es en toi sans changement, tu fixes dans leur succession les temps que le soleil mesure.\\
\colored{D}onne-nous un soir lumineux où la vie ne décline pas, où, pour fruit de la Sainte Mort, resplendira toujours la gloire.\\
\colored{E}xauce-nous, Père très bon, et toi, le Fils égal au Père, avec l’Esprit Consolateur, régnant pour les siècles des siècles.}

\gscore{an_non_vos_me_elegistis}
\translation{\aa Ce n'est pas vous qui m'avez choisi, c'est moi qui vous ai choisis, pour que vous portiez du fruit, et que votre fruit demeure.}
\smalltitle{Psaume 125}
\psalm{125}{4}
\smalltitle{Psaume 126}
\psalm{126}{4}
\smalltitle{Psaume 127}
\psalm{127}{4}

\capitulum{Ph 4, 8. 9b}
{Fratres, quæcúmque sunt vera, quæcúmque pudíca, quæcúmque
iusta, quæcúmque casta, † quæcúmque amabília, quæcúmque bonæ
famæ, si qua virtus et si qua laus, hæc cogitáte ; * et Deus pacis erit
vobíscum.}
{Enfin, mes frères, tout ce qui est vrai et noble, tout ce qui est juste et pur,
tout ce qui est digne d’être aimé et honoré, tout ce qui s’appelle vertu
et qui mérite des éloges, tout cela, prenez-le en compte. Et le Dieu de la
paix sera avec vous.}

\versiculus{Cantábo tibi, Dómine}{Psallam et intéllegam in via immaculáta}{À toi mes hymnes, Seigneur}{Je chanterai en suivant le chemin le plus parfait}

\rubric{Oraison des Vêpres, p. \pageref{0723V}.}

\gscore{or_benedicamus_hm_festivus}

\bigtitle{Sainte Brigitte, aux Vêpres}{mardi 23 juillet}{Vêpres}

\smallscore{or_dia_festivus}

\gscore{hy_fortem_virili_pectore}
\translation{\colored{L}ouons tous cette femme forte au cœur généreux ; la gloire de sa sainteté la rend illustre en tous lieux.\\
\colored{B}lessée par l’amour divin, elle a foulé aux pieds les biens éphémères du siècle pour parcourir le chemin escarpé du ciel.\\
\colored{E}n domptant sa chair par le jeûne et en nourrissant son âme des délices de la prière, elle a acquis les joies célestes.\\
\colored{C}hrist Roi, force des forts, unique auteur de la grandeur, par l’intercession de cette sainte, écoute avec bonté nos supplications.\\
\colored{G}loire à toi, Jésus, qui nous donnes d’espérer les suffrages de la bienheureuse servante et les récompenses éternelles.}

\smalltitle{Psaume 109}
\gscore{an_accinxit_fortitudine}
\translation{\aa Elle s'est ceinte de force, et a rendu fort son bras: c'est pouquoi sa lampe ne s'éteindra pas, éternellement.}
\psalm{109}{8}

\smalltitle{Psaume 112}
\gscore{an_via_iustorum}
\translation{\aa Le chemin des justes a été rendu droit, et le sentier des saints a été préparé.}
\psalm{112}{8}

\smalltitle{Psaume 121}
\gscore{an_cognovit_eam}
\translation{\aa Le Seigneur l'a connue dans ses bénédictions, et elle a trouvé grâce auprès du Seigneur.}
\psalm{121}{8}

\smalltitle{Psaume 126}
\gscore{an_lauda_qui_posuit}
\translation{\aa Célèbre le Seigneur, Jérusalem, car il a mis la paix à tes frontières, et en toi il a béni tes fils.}
\psalm{126}{7}

\capitulum{Rm 8, 28-30}
{Scimus quóniam diligéntibus Deum ómnia cooperántur in bonum, *
his qui secúndum propósitum vocáti sunt. Nam, quos præscívit,
et prædestinávit confórmes fíeri imáginis Fílii eius, * ut sit ipse
primogénitus in multis frátribus ; / quos autem prædestinávit, hos et
vocávit ; † et quos vocávit, hos et iustificávit ; * quos autem iustificávit,
illos et glorificávit.}
{Nous le savons, quand les hommes aiment Dieu, lui-même fait tout
contribuer à leur bien, puisqu’ils sont appelés selon le dessein de son
amour. Ceux que, d’avance, il connaissait, il les a aussi destinés d’avance
à être configurés à l’image de son Fils, pour que ce Fils soit le premierné
d’une multitude de frères. Ceux qu’il avait destinés d’avance, il les a
aussi appelés ; ceux qu’il a appelés, il en a fait des justes ; et ceux qu’il
rendus justes, il leur a donné sa gloire.}

\gscore{rb_elegit_eam}
\translation{\rr Dieu l'a choisie, il l'a prédestinée.\\
\vv Il l'a fait habiter dans sa demeure.}

\newpage

\smalltitle{Magnificat}
\gscore{an_ego_dominus_sponsabo_te}
\translation{\aa Moi, le Seigneur, je te fiancerai à moi pour toujours,
je te fiancerai à moi dans la justice et le droit, dans la fidélité et la tendresse ;
je te fiancerai à moi dans la loyauté.}
\incipit{magn}{7dsol}
\psalm{magn}{7sol}

\smalltitle{Intercession}
\smallscore{or_kyrie_festivus}
\smallscore{or_pater_festivus}

\needspace{2cm}
\label{0723V}
\oratio{
Deus, qui beátam Birgíttam per várias vitae sémitas duxísti,
eámque sapiéntiam crucis in contemplatióne passiónis Fílii tui
mirabíliter docuísti, † concéde nobis, ut, digne in tua vocatióne
ambulántes, * te in ómnibus quæ´ rere valeámus. Per Dóminum.
}{
Seigneur Dieu, tu as conduit sainte Brigitte par divers chemins de vie,
et tu lui as enseigné de façon admirable la sagesse de la croix par la
contemplation de la Passion de ton Fils ; accorde à chacun de nous, quel
que soit son état de vie, de savoir te chercher en toute chose.
}

\blessing

\gscore{or_benedicamus_festis}
%%% fin en fin de page de gauche, reprise en page de droite
\end{document}
