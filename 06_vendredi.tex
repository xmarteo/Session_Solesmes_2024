% !TEX TS-program = lualatex
% !TEX encoding = UTF-8

\documentclass[Session2024.tex]{subfiles}

\ifcsname preamble@file\endcsname
  \setcounter{page}{\getpagerefnumber{M-06_vendredi}}
\fi

\begin{document}

\bigtitle{Saints Anne et Joachim, à la Messe}{vendredi 26 juillet}{Messe}

\gscore{in_sapientiam_sanctorum}
\translation{Que les peuples racontent la sagesse des saints, et que l'assemblée publie leurs louanges : leur nom vivra de génération en génération.\\
\vv Justes, réjouissez-vous dans le Seigneur; c'est aux hommes droits que sied la louange.
TODO versets?}

\smalltitle{Psalmodie de Tierce}
\gscore{an_iusti_confitebuntur}
\translation{Les justes confesseront ton nom, et les coeurs droits vivront en ta présence.}
\smalltitle{Psaume 119}
\psalm{119}{1}
\smalltitle{Psaume 120}
\psalm{120}{1}
\smalltitle{Psaume 121}
\psalm{121}{1}

\gscore{ky_k7adlib}
\gscore{gr_exsultabunt_sancti}
\translation{\rr Les saints tressailliront dans la gloire ; ils se réjouiront sur leurs couches.\\
\vv Chantez au Seigneur un cantique nouveau : que Sa louange retentisse dans l’assemblée des saints.}
\gscore{al_o_ioachim}
\translation{\vv Ô Joachim, saint époux d’Anne, père de la Vierge nourricière, aidez ici-bas au salut de vos serviteurs.}
\gscore{of_laetamini_in_domino}
\translation{Justes, réjouissez-vous dans le Seigneur, et soyez dans l'allégresse; et glorifiez-vous en lui, vous tous qui avez le cœur droit.}
\gscore{ky_s2adlib}
\gscore{ky_a2adlib}
\gscore{co_ierusalem_surge}
\translation{Lève-toi , Jérusalem, et monte sur un lieu élevé : et considère les délices que ton Dieu versera en toi. TODO versets}

\end{document}