% !TEX TS-program = lualatex
% !TEX encoding = UTF-8

\documentclass[Session2024.tex]{subfiles}

\ifcsname preamble@file\endcsname
  \setcounter{page}{\getpagerefnumber{M-04_mercredi}}
\fi

\begin{document}

\bigtitle{Mercredi de la 16\textsuperscript{e} semaine \emph{per annum}, à la Messe}{mercredi 24 juillet}{Messe}

\gscore{in_ecce_deus_adiuvat}
\translation{Voici que Dieu vient à mon aide, et que le Seigneur est le protecteur de ma vie. Fais retomber les maux sur mes ennemis, et extermine-les dans ta vérité.\\
\vv Ô Dieu, sauve-moi par ton nom, et rends-moi justice par ta puissance.
TODO versets?}

\smalltitle{Psalmodie de Tierce}
\gscore{an_clamavi}
\translation{TODO}
\smalltitle{Psaume 119}
\psalm{119}{E}
\smalltitle{Psaume 120}
\psalm{120}{E}
\smalltitle{Psaume 121}
\psalm{121}{E}

\gscore{ky_k16}
\gscore{gr_domine_dominus_noster}
\translation{\rr Seigneur, notre Maître, que Ton Nom est admirable dans toute la terre !\\
\vv Car Ta magnificence est élevée au-dessus des cieux.}
\gscore{al_eripe_me_de_inimicis}
\translation{\vv Sauve-moi des mains de mes ennemis, ô mon Dieu, et délivre-moi de ceux qui se lèvent contre moi.}
\gscore{of_iustitiae_domini}
\translation{Les justices du Seigneur sont droites, elles réjouissent les coeurs ; et ses préceptes sont plus doux que le miel et qu'un rayon plein de miel. Aussi ton serviteur les observe.}
\gscore{ky_s16}
\gscore{ky_a16}
\gscore{co_acceptabis_sacrificium}
\translation{Tu agréeras un sacrifice de justice, les oblations et les holocaustes sur ton autel. TODO versets}

\bigtitle{Mercredi IV, à Sexte}{mercredi 24 juillet}{Sexte}

\smallscore{or_dia_ferialis}

\gscore{hy_rector_potens_ferialis}
\translation{\colored{M}aître puissant, Dieu Vérité, tu règles la marche du temps, tu formes l’aube en sa clarté, au midi tu donnes ses flammes.\\
\colored{E}teins le feu des dissensions, calme la fièvre du péché, apporte à nos corps la santé, à nos cœurs, la paix véritable.\\
\colored{E}xauce-nous, Père très bon, et toi, le Fils égal au Père, avec l’Esprit Consolateur, régnant pour les siècles des siècles.}

\gscore{an_qui_habitas}
\translation{TODO}
\smalltitle{Psaume 122}
\psalm{122}{8}
\smalltitle{Psaume 123}
\psalm{123}{8}
\smalltitle{Psaume 124}
\psalm{124}{8}

\capitulum{Col 3, 17}
{Omne quodcúmque fácitis in verbo aut in ópere, † ómnia in nómine
Dómini Iesu * grátias agéntes Deo Patri per ipsum.}
{Tout ce que vous dites, tout ce que vous faites, que ce soit toujours au
nom du Seigneur Jésus-Christ, en offrant par lui votre action de grâce à
Dieu le Père.}

\versiculus{Tibi, Dómine, sacrificábo hóstiam laudis}{Et nomen Dómini invocábo}{Je t’offrirai, Seigneur, le sacrifice d’action de grâce}{J’invoquerai le nom du Seigneur}

\oratio{Omnípotens et miséricors Deus, qui nos die média
respiráre concédis, + quos cœ́pimus propítius intuére
labóres, * et, sanans quæ delíquimus, fac eos ad finem
tibi plácitum perveníre. Per Christum.}
{Père, au milieu du jour tu nous
donnes un temps de repos
pour refaire nos corps et nos
esprits, accorde-nous de le recevoir
dans la reconnaissance et
d’en tirer profit pour ton service
et celui de nos frères. Par
le Christ.}

\gscore{or_benedicamus_hm_ferialis}

\bigtitle{Mercredi IV, à None}{mercredi 24 juillet}{None}

\smallscore{or_dia_ferialis}

\gscore{hy_rerum_deus_ferialis}
\translation{\colored{D}ieu fort, soutien de l’univers, qui es en toi sans changement, tu fixes dans leur succession les temps que le soleil mesure.\\
\colored{D}onne-nous un soir lumineux où la vie ne décline pas, où, pour fruit de la Sainte Mort, resplendira toujours la gloire.\\
\colored{E}xauce-nous, Père très bon, et toi, le Fils égal au Père, avec l’Esprit Consolateur, régnant pour les siècles des siècles.}

\gscore{an_beati_omnes}
\translation{}
\smalltitle{Psaume 125}
\psalm{125}{2}
\smalltitle{Psaume 126}
\psalm{126}{2}
\smalltitle{Psaume 127}
\psalm{127}{2}

\capitulum{Col 3, 23-24}
{Quodcúmque fácitis, ex ánimo operámini sicut Dómino et non
homínibus, * sciéntes quod a Dómino accipiétis retributiónem hereditátis. / Dómino Christo servíte.}
{Quel que soit votre travail, faites-le de bon coeur, pour le Seigneur et non
pour plaire à des hommes : vous savez bien qu’en retour le Seigneur fera
de vous ses héritiers. Le maître c’est le Christ : vous êtes à son service.}

\versiculus{Dóminus pars hereditátis meæ et cálicis mei}{Tu es qui détines sortem meam}{Seigneur, mon partage et ma coupe}{De toi dépend mon sort}

\oratio{Dómine Iesu Christe, qui manus tuas in cruce ad
salvándos hómines extendísti, + concéde, ut actus nostri
tibi reddántur accépti, * et opus tuæ redemptiónis in
mundo váleant declaráre. Qui vivis.}{Seigneur Jésus Christ, toi qui
étendis les bras sur la croix
pour sauver tous les hommes,
donne-nous de te plaire en chacun
de nos actes pour faire
connaître au monde l’œuvre de
ton amour. Toi qui.}

\gscore{or_benedicamus_hm_ferialis}

\bigtitle{Mercredi IV, aux Vêpres}{mercredi 24 juillet}{Vêpres}

\smallscore{or_dia_ferialis}

\gscore{hy_sol_ecce_lentus_occidens}
\translation{\colored{V}oici que le soleil se couche lentement, qu’il abandonne avec tristesse les monts, les labours et les mers, mais renouvelle le présage de la lumière de demain.\\
\colored{P}our l’émerveillement des hommes, Créateur, dans ta Providence, tu as ainsi posé les lois qui font alterner dans le temps et les ombres et la lumière.\\
\colored{E}t au moment où le silence des ténèbres oppresse le ciel, quand les forces nous manquent pour travailler, quand nous cherchons un repos désiré,\\
\colored{R}iches alors d’espérance et de foi, nous sommes comblés par la lumière de ton Verbe, qui est, depuis les siècles, éclat de la gloire du Père.\\
\colored{I}l est le soleil qui ne connaît jamais de lever, ni de soir ; la terre désire être vêtue de sa lumière ; les cieux tiennent de lui leur éternelle joie\\
\colored{E}t donne-nous de jouir pour toujours de cette lumière sereine : à toi, au Fils, à l’Esprit Saint, nous dédierons des chants nouveaux.}

\smalltitle{Psaume 134}
\gscore{an_omnia_quaecumque}
\translation{TODO}
\psalm{134}{3}

\smalltitle{Psaume 135}
\gscore{an_quoniam_in_aeternum}
\translation{TODO}
\psalm{135}{3}

\smalltitle{Psaume 136}
\gscore{an_hymnum_cantate}
\translation{TODO}
\psalm{136}{8}

\smalltitle{Psaume 137}
\gscore{an_in_conspectu_angelorum}
\translation{TODO}
\psalm{137}{5}

\capitulum{1 Jn 2, 3-6}
{In hoc cognóscimus quóniam nóvimus Christum : * si mandáta eius
servémus. / Qui dicit : « Novi eum », et mandáta eius non servat, mendax
est, et in isto véritas non est ; † qui autem servat verbum eius, * vere in
hoc cáritas Dei consummáta est. / In hoc cognóscimus quóniam in ipso
sumus. / Qui dicit se in ipso manére, debet, sicut ille ambulávit, et ipse
ambuláre.}
{Voici comment nous pouvons savoir que nous connaissons Jésus-Christ :
c’est en gardant ses commandements. Celui qui dit : « Je le connais »,
et qui ne garde pas ses commandements, est un menteur : la vérité n’est
pas en lui. Mais en celui qui garde fidèlement sa parole, l’amour de Dieu
atteint vraiment la perfection : voilà comment nous reconnaissons que
nous sommes en lui. Celui qui déclare demeurer en lui doit marcher luimême
dans la voie où lui, Jésus, a marché.}

\gscore{rb_redime_me_domine}
\translation{TODO}

\smalltitle{Magnificat}
\gscore{an_fac_deus_potentiam}
\translation{TODO}
\incipit{magn}{7c}
\psalm{magn}{7}

\smalltitle{Intercessions}
\smallscore{or_kyrie_ferialis}
\smallscore{or_pater_ferialis}

\oratio{Recordáre, Dómine, misericórdiæ tuæ, ut qui
esuriéntes bonis cæléstibus implére dignáris,
indigéntiæ nostræ tríbuas tuis abundáre divítiis. Per
Dóminum.}{Souviens-toi, Seigneur, de ton
amour, afin qu’ayant comblé
les affamés des biens du ciel, tu
accordes aussi aux pauvres que
nous sommes l’abondance de tes
richesses. Par Jésus Christ.}

\blessing

\gscore{or_benedicamus_feriis}

\end{document}
