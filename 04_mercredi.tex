% !TEX TS-program = lualatex
% !TEX encoding = UTF-8

\documentclass[Session2024.tex]{subfiles}

\ifcsname preamble@file\endcsname
  \setcounter{page}{\getpagerefnumber{M-04_mercredi}}
\fi

\begin{document}

\bigtitle{Mercredi de la 16\textsuperscript{e} semaine \emph{per annum}, à la Messe}{mercredi 24 juillet}{Messe}

\gscore{in_ecce_deus_adiuvat}
\translation{TODO}

\smalltitle{Psalmodie de Tierce}
\gscore{an_clamavi}
\translation{TODO}
\psalm{119}{E}
\psalm{120}{E}
\psalm{121}{E}

\gscore{ky_k16}
\gscore{gr_domine_dominus_noster}
\translation{TODO}
\gscore{al_eripe_me_de_inimicis}
\translation{TODO}
\gscore{of_iustitiae_domini}
\translation{TODO}
\gscore{ky_s16}
\gscore{ky_a16}
\gscore{co_acceptabis_sacrificium}
\translation{TODO}

\bigtitle{Mercredi IV, à Sexte}{mercredi 24 juillet}{Sexte}

\gscore{or_dia_ferialis}

\gscore{hy_rector_potens_ferialis}

\gscore{an_qui_habitas}
\translation{TODO}
\psalm{122}{8}
\psalm{123}{8}
\psalm{124}{8}

\capitulus{Col 3, 17}
{Omne quodcúmque fácitis in verbo aut in ópere, † ómnia in nómine
Dómini Iesu * grátias agéntes Deo Patri per ipsum.}
{Tout ce que vous dites, tout ce que vous faites, que ce soit toujours au
nom du Seigneur Jésus-Christ, en offrant par lui votre action de grâce à
Dieu le Père.}

\versiculus{Tibi, Dómine, sacrificábo hóstiam laudis}{Et nomen Dómini invocábo}{Je t’offrirai, Seigneur, le sacrifice d’action de grâce}{J’invoquerai le nom du Seigneur}

\oratio{Omnípotens et miséricors Deus, qui nos die média
respiráre concédis, + quos cœ́pimus propítius intuére
labóres, * et, sanans quæ delíquimus, fac eos ad finem
tibi plácitum perveníre. Per Christum.}
{Père, au milieu du jour tu nous
donnes un temps de repos
pour refaire nos corps et nos
esprits, accorde-nous de le recevoir
dans la reconnaissance et
d’en tirer profit pour ton service
et celui de nos frères. Par
le Christ.}

\gscore{or_benedicamus_hm_ferialis}

\bigtitle{Mercredi IV, à None}{mercredi 24 juillet}{None}

\gscore{or_dia_ferialis}

\gscore{hy_rerum_deus_ferialis}
\translation{TODO}

\gscore{an_beati_omnes}
\translation{}
\psalm{125}{2}
\psalm{126}{2}
\psalm{127}{2}

\capitulus{Col 3, 23-24}
{Quodcúmque fácitis, ex ánimo operámini sicut Dómino et non
homínibus, * sciéntes quod a Dómino accipiétis retributiónem hereditátis. / Dómino Christo servíte.}
{Quel que soit votre travail, faites-le de bon coeur, pour le Seigneur et non
pour plaire à des hommes : vous savez bien qu’en retour le Seigneur fera
de vous ses héritiers. Le maître c’est le Christ : vous êtes à son service.}

\versiculus{Dóminus pars hereditátis meæ et cálicis mei}{Tu es qui détines sortem meam}{Seigneur, mon partage et ma coupe}{De toi dépend mon sort}

\oratio{Dómine Iesu Christe, qui manus tuas in cruce ad
salvándos hómines extendísti, + concéde, ut actus nostri
tibi reddántur accépti, * et opus tuæ redemptiónis in
mundo váleant declaráre. Qui vivis.}{Seigneur Jésus Christ, toi qui
étendis les bras sur la croix
pour sauver tous les hommes,
donne-nous de te plaire en chacun
de nos actes pour faire
connaître au monde l’œuvre de
ton amour. Toi qui.}

\gscore{or_benedicamus_hm_ferialis}

\bigtitle{Mercredi IV, aux Vêpres}{mercredi 24 juillet}{Vêpres}

\gscore{or_dia_ferialis}

\gscore{hy_sol_ecce_lentus_occidens}
\translation{}

\gscore{an_omnia_quaecumque}
\translation{TODO}
\psalm{134}{3}

\gscore{an_quoniam_in_aeternum}
\translation{TODO}
\psalm{135}{3}

\gscore{an_hymnum_cantate}
\translation{TODO}
\psalm{136}{8}

\gscore{an_in_conspectu_angelorum}
\translation{TODO}
\psalm{137}{5}

\capitulus{1 Jn 2, 3-6}
{In hoc cognóscimus quóniam nóvimus Christum : * si mandáta eius
servémus. / Qui dicit : « Novi eum », et mandáta eius non servat, mendax
est, et in isto véritas non est ; † qui autem servat verbum eius, * vere in
hoc cáritas Dei consummáta est. / In hoc cognóscimus quóniam in ipso
sumus. / Qui dicit se in ipso manére, debet, sicut ille ambulávit, et ipse
ambuláre.}
{Voici comment nous pouvons savoir que nous connaissons Jésus-Christ :
c’est en gardant ses commandements. Celui qui dit : « Je le connais »,
et qui ne garde pas ses commandements, est un menteur : la vérité n’est
pas en lui. Mais en celui qui garde fidèlement sa parole, l’amour de Dieu
atteint vraiment la perfection : voilà comment nous reconnaissons que
nous sommes en lui. Celui qui déclare demeurer en lui doit marcher luimême
dans la voie où lui, Jésus, a marché.}

\gscore{rb_redime_me_domine}
\translation{TODO}

\gscore{an_fac_deus_potentiam}
\translation{TODO}
\incipit{magn}{7c}
\psalm{magn}{7}

\gscore{or_kyrie_ferialis}
\gscore{or_pater_ferialis}

\oratio{Recordáre, Dómine, misericórdiæ tuæ, ut qui
esuriéntes bonis cæléstibus implére dignáris,
indigéntiæ nostræ tríbuas tuis abundáre divítiis. Per
Dóminum.}{Souviens-toi, Seigneur, de ton
amour, afin qu’ayant comblé
les affamés des biens du ciel, tu
accordes aussi aux pauvres que
nous sommes l’abondance de tes
richesses. Par Jésus Christ.}

\gscore{or_benedictio_ferialis}

\gscore{or_benedicamus_feriis}

\end{document}
