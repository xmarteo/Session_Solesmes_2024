% !TEX TS-program = lualatex
% !TEX encoding = UTF-8

\documentclass[Session2024.tex]{subfiles}

\ifcsname preamble@file\endcsname
  \setcounter{page}{\getpagerefnumber{M-02_lundi}}
\fi

\begin{document}

\bigtitle{Sainte Marie Madeleine, à la Messe}{lundi 22 juillet}{Messe}

\gscore{in_tibi_dixit}
\translation{TODO}

\smalltitle{Psalmodie de Tierce}
\gscore{an_dum_transisset}
\translation{Le sabbat terminé, Marie Madeleine, Marie, mère de Jacques, et Salomé
achetèrent des parfums
pour aller embaumer le corps de Jésus, alléluia.}
\smalltitle{Psaume 118, xiii}
\psalm{118-13}{4}
\smalltitle{Psaume 118, xiv}
\psalm{118-14}{4}
\smalltitle{Psaume 118, xv}
\psalm{118-15}{4}
\smalltitle{Psaume 118, xvi}
\psalm{118-16}{4}

\gscore{ky_k4}
\gscore{ky_g4}
\gscore{gr_audi_filia}
\translation{TODO}
\gscore{al_surrexit_dominus_de_sepulcro}
\translation{TODO}
\gscore{of_deus_deus_meus}
\translation{TODO}
\gscore{ky_s4}
\gscore{ky_a4}
\gscore{co_notas_mihi_fecisti}
\translation{TODO}

\bigtitle{Sainte Marie Madeleine, à Sexte}{lundi 22 juillet}{Sexte}

\smallscore{or_dia_ferialis}

\gscore{hy_rector_potens_festivus}
\translation{TODO}

\gscore{an_ardens_est_cor}
\translation{Le coeur brûlant, je désire voir mon Seigneur ;
je le cherche, et je ne trouve pas
où ils l’ont déposé, alléluia.}
\smalltitle{Psaume 118, xvii}
\psalm{118-17}{1}
\smalltitle{Psaume 118, xvii}
\psalm{118-18}{1}
\smalltitle{Psaume 118, xix}
\psalm{118-19}{1}

\capitulum{Ct 3, 3-4}
{Num, quem díligit ánima mea, vidístis ? Páululum cum
pertransíssem eos, † invéni, quem díligit ánima mea. * Ténui eum, nec
dimíttam.}
{Celui que mon âme désire, l’auriez-vous vu ? À peine les avais-je
dépassés, j’ai trouvé celui que mon âme désire : je l’ai saisi et ne le
lâcherai pas.}

\versiculus{Invéni quem díligit ánima mea}{Ténui eum, nec dimíttam}{J’ai trouvé celui que mon coeur aime}{Je l’ai saisi, je ne le lâcherai pas}

\rubric{Oraison des Vêpres, p. \pageref{0722V}.}

\gscore{or_benedicamus_hm_festivus}

\bigtitle{Sainte Marie Madeleine, à None}{lundi 22 juillet}{None}

\smallscore{or_dia_ferialis}

\gscore{hy_rerum_deus_festivus}
\translation{TODO}

\gscore{an_dicit_iesus_maria}
\translation{Jésus dit alors : « Marie ! » Elle se tourne vers lui et lui dit :
« Rabbouni ! » ce qui veut dire : « Maître ».}
\smalltitle{Psaume 118, xx}
\psalm{118-20}{4}
\smalltitle{Psaume 118, xxi}
\psalm{118-21}{4}
\smalltitle{Psaume 118, xxii}
\psalm{118-22}{4}

\capitulum{Ct 5, 5}
{Surréxi, ut aperírem dilécto meo ; † manus meæ stillavérunt
myrrham, * et dígiti mei pleni myrrha probatíssima super ansam
pessúli.}
{Je me suis levée pour ouvrir à mon bien-aimé, les mains ruisselantes
de myrrhe. Mes doigts répandaient cette myrrhe sur la barre du verrou.}

\versiculus{Cantábo tibi, Dómine}{Psallam et intéllegam in via immaculáta}{À toi mes hymnes, Seigneur}{Je chanterai en suivant le chemin le plus parfait}

\rubric{Oraison des Vêpres, p. \pageref{0722V}.}

\gscore{or_benedicamus_hm_festivus}

\bigtitle{Sainte Marie Madeleine, aux Vêpres}{lundi 22 juillet}{Vêpres}

\smallscore{or_dia_festivus}

\rubric{À la dernière strophe de cette hymne, on ne s'incline pas.}
\gscore{hy_magdalae_sidus}
\translation{Étoile de Magdala, heureuse femme, nous te vénérons tous,
toi que le Christ s’est associée par le noeud d’un étroit amour.\\\\
Quand se révèle à toi sa puissance
redoutable qui chasse les forces démoniaques,
pleine de reconnaissance
pour celui qui t’a guérie, tu te réjouis
d’être maintenant enchaînée par le
lien plus fort de la foi.\\\\
Dès lors, ton amour te presse de te
tenir aux pieds du Maître ; tu l’accompagnes
et l’entoures avec ferveur de
soins empressés.\\\\
Avec celles qui pleurent le Seigneur,
debout au pied de la croix, tu brûles
d’une ardente tendresse ; tu laves avec
amour et tu embaumes ses membres
avant de les livrer au tombeau.\\\\
L’amour du Christ nous a engendrés
: fais que nous soyons unis à ton
triomphe pour l’éternité, et que nous
chantions avec toi à profusion les
louanges du Bien-Aimé.}

\smalltitle{Psaume 109}
\gscore{an_dum_transisset}
\translation{Le sabbat terminé, Marie Madeleine, Marie, mère de Jacques, et Salomé
achetèrent des parfums
pour aller embaumer le corps de Jésus, alléluia.}
\psalm{109}{4}

\smalltitle{Psaume 112}
\gscore{an_inclinavit_se_maria}
\translation{Marie se pencha vers l’intérieur du tombeau,
et, elle aperçut deux anges,
assis, vêtus de blanc, alléluia.}
\psalm{112}{1}

\smalltitle{Psaume 121}
\gscore{an_ardens_est_cor}
\translation{Le coeur brûlant, je désire voir mon Seigneur ;
je le cherche, et je ne trouve pas
où ils l’ont déposé, alléluia.}
\psalm{121}{8}

\smalltitle{Psaume 126}
\gscore{an_dicit_iesus_maria}
\translation{Jésus dit alors : « Marie ! » Elle se tourne vers lui et lui dit :
« Rabbouni ! » ce qui veut dire : « Maître ».}
\psalm{126}{4}

\capitulum{Rm 8, 28-30}
{Scimus quóniam diligéntibus Deum ómnia cooperántur in bonum, *
his qui secúndum propósitum vocáti sunt. Nam, quos præscívit,
et prædestinávit confórmes fíeri imáginis Fílii eius, * ut sit ipse
primogénitus in multis frátribus ; / quos autem prædestinávit, hos et
vocávit ; † et quos vocávit, hos et iustificávit ; * quos autem iustificávit,
illos et glorificávit.}
{Nous le savons, quand les hommes aiment Dieu, lui-même fait tout
contribuer à leur bien, puisqu’ils sont appelés selon le dessein de son
amour. Ceux que, d’avance, il connaissait, il les a aussi destinés d’avance
à être configurés à l’image de son Fils, pour que ce Fils soit le premierné
d’une multitude de frères. Ceux qu’il avait destinés d’avance, il les a
aussi appelés ; ceux qu’il a appelés, il en a fait des justes ; et ceux qu’il
rendus justes, il leur a donné sa gloire.}

\gscore{rb_tibi_dixit}
\translation{Mon coeur t’a dit : J’ai recherché ta face. C’est ta face, Seigneur, que je cherche.}

\smalltitle{Magnificat}
\gscore{an_venit_maria_nuntians}
\translation{Marie Madeleine s’en va annoncer aux disciples :
« J’ai vu le Seigneur, alléluia. »}
\incipit{magn}{7asol}
\psalm{magn}{7sol}

\smalltitle{Intercessions}
\smallscore{or_kyrie_festivus}
\smallscore{or_pater_festivus}

\label{0722V}
\oratio{
Deus, cuius Unigénitus Maríæ Magdalénæ ante omnes gáudium
nuntiándum paschále commísit, † præsta, quæ´ sumus, ut, eius
intercessióne et exémplo, * Christum vivéntem prædicémus, et in
glória tua regnántem videámus. Qui tecum.
}{
Seigneur Dieu, c’est à Marie Madeleine que ton Fils unique a confié la
première annonce de la joie pascale ; accorde-nous, à sa prière et à son
exemple, de proclamer que le Christ est vivant et de le contempler dans
la gloire de ton Royaume.
}

\smallscore{or_benedictio_festivus}

\gscore{or_benedicamus_festis}

\end{document}