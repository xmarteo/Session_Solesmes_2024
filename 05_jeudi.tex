% !TEX TS-program = lualatex
% !TEX encoding = UTF-8

\documentclass[Session2024.tex]{subfiles}

\ifcsname preamble@file\endcsname
  \setcounter{page}{\getpagerefnumber{M-05_jeudi}}
\fi

\begin{document}

\bigtitle{Saint Jacques, à la Messe}{jeudi 25 juillet}{Messe}

\gscore{in_mihi_autem_nimis_honorati}
\translation{TODO}

\smalltitle{Psalmodie de Tierce}
\gscore{an_dum_perambularet}
\translation{Comme le Seigneur marchait
le long de la mer de Galilée,
il vit Jacques et Jean, et il les appela.}
\psalm{119}{1}
\psalm{120}{1}
\psalm{121}{1}

\gscore{ky_k4}
\gscore{ky_g4}
\gscore{gr_constitues_eos}
\translation{TODO}
\gscore{al_ego_vos_elegi}
\translation{TODO}
\gscore{of_in_omnem_terram}
\translation{TODO}
\gscore{ky_s4}
\gscore{ky_a4}
\gscore{co_ego_vos_elegi}
\translation{TODO}

\bigtitle{Saint Jacques, à Sexte}{jeudi 25 juillet}{Sexte}

\gscore{or_dia_ferialis}

\gscore{hy_rector_potens_festivus}

\gscore{an_sedere_autem}
\translation{Siéger à ma droite et à ma gauche,
ce n’est pas à moi de l’accorder ;
il y a ceux pour qui cela est préparé par mon Père.}
\psalm{122}{7}
\psalm{123}{7}
\psalm{124}{7} % TODO on peut renvoyer au mardi pour cette psalmodie

\capitulum{Ac 5, 12a. 14}
{Per manus apostolórum fiébant signa et prodígia multa in plebe. †
Magis autem addebántur credéntes Dómino, * multitúdines virórum ac
mulíerum.}
{Par les mains des Apôtres, beaucoup de signes et de prodiges s’accomplissaient
dans le peuple. De plus en plus, des foules d’hommes et de
femmes, en devenant croyants, s’attachaient au Seigneur.}

\versiculus{Custodiébant testimónia Dei}{Et præcépta eius}{Ils ont gardé les volontés du Seigneur}{Les lois qu’il leur donna}

\rubric{Oraison des Vêpres, p. \pageref{0725V}.}

\gscore{or_benedicamus_hm_festivus}

\bigtitle{Saint Jacques, à None}{jeudi 25 juillet}{None}

\gscore{or_dia_ferialis}

\gscore{hy_rerum_deus_festivus}
\translation{TODO}

\gscore{an_occidit_herodes}
\translation{Hérode fit tuer Jacques,
frère de Jean, par le glaive.}
\psalm{125}{8}
\psalm{126}{8}
\psalm{127}{8}

\capitulum{Ac 5, 41-42}
{Ibant apóstoli gaudéntes a conspéctu concílii, † quóniam digni hábiti
sunt pro nómine contuméliam pati ; * et omni die in templo et circa
domos non cessábant docéntes et evangelizántes Christum, Iesum.}
{Les Apôtres, quittant le Conseil suprême, repartaient tout joyeux d’avoir
été jugés dignes de subir des humiliations pour le nom de Jésus. Tous
les jours, au Temple et dans leurs maisons, sans cesse, ils enseignaient
et annonçaient la Bonne Nouvelle : le Christ, c’est Jésus.}

\versiculus{Gaudéte et exsultáte, dicit Dóminus}{Quia nómina vestra scripta sunt in cælis}{Réjouissez-vous, exultez, dit le Seigneur}{Vos noms sont inscrits dans les cieux}

\rubric{Oraison des Vêpres, p. \pageref{0725V}.}

\gscore{or_benedicamus_hm_festivus}

\bigtitle{Saint Jacques, aux Vêpres}{jeudi 25 juillet}{Vêpres}

\gscore{or_dia_festivus}

\gscore{hy_exsultet_caelum_laudibus}
\translation{Que la louange exulte au ciel, que la joie réponde sur terre ! car c’est la
gloire des Apôtres que nous célébrons aujourd’hui.

Vous, les justes juges des hommes
Et les vraies lumières du monde,
Voici les voeux de notre coeur :
Écoutez nos voix suppliantes.

Vous qui pouvez fermer le ciel
Et délier pour nous ses portes,
Nous vous prions : dites le mot
Qui nous délie de tout péché.

Puisque santé et maladie
Obéissent à vos paroles,
Guérissez notre coeur malade,
À notre âme rendez vigueur.

Ainsi, quand reviendra le Christ
Pour juger, à la fin des temps,
Il nous fera participer
Au bonheur qui n’a pas de fin.}

\gscore{an_isti_sunt_viri_sancti}
\translation{TODO}
\psalm{109}{7}

\gscore{an_vos_estis}
\translation{TODO}
\psalm{112}{2}

\gscore{an_isti_viventes_in_carne}
\translation{TODO}
\psalm{115}{7}

\gscore{an_iam_non_dicam_vos_servos}
\translation{TODO}
\psalm{125}{1d}

\capitulum{Ep 4, 11-13}
{Christus dedit quosdam quidem apóstolos, quosdam autem
prophétas, † álios vero evangelístas, álios autem pastóres et doctóres *
ad instructiónem sanctórum in opus ministérii, in ædificatiónem córporis
Christi, † donec occurrámus omnes in unitátem fídei et agnitiónis Fílii
Dei, * in virum perféctum, in mensúram ætátis plenitúdinis Christi.}
{Les dons que le Christ a faits, ce sont les Apôtres, et aussi les prophètes,
les évangélisateurs, les pasteurs et ceux qui enseignent. De cette
manière, les fidèles sont organisés pour que les tâches du ministère
soient accomplies et que se construise le corps du Christ, jusqu’à ce
que nous parvenions tous ensemble à l’unité dans la foi et la pleine
connaissance du Fils de Dieu, à l’état de l’Homme parfait, à la stature
du Christ dans sa plénitude.}

\gscore{rb_annuntiate_inter_gentes}
\translation{Racontez à tous les peuples la gloire du Seigneur, À toutes les nations ses merveilles}

\gscore{an_quicumque_voluerit_inter_vos}
\translation{Celui qui veut devenir grand parmi vous sera votre serviteur ;
et celui qui veut être parmi vous le premier sera votre esclave.}
\incipit{magn}{8gsol}
\psalm{magn}{8sol}

\gscore{or_kyrie_festivus}
\gscore{or_pater_festivus}

\label{0725V}
\oratio{
Omnípotens sempitérne Deus, qui Apostolórum tuórum primítias
beáti Iacóbi sánguine dedicásti, † da, quǽsumus, Ecclésiæ tuæ ipsíus
confessióne firmári, * et iúgiter patrocíniis confovéri. Per Dóminum.
}{
Dieu éternel et tout-puissant, tu as consacré l’offrande du bienheureux
Jacques, le premier de tes Apôtres à verser pour toi son sang, accorde à
ton Église de trouver dans son témoignage une force, et dans sa protection
un appui constant.
}

\gscore{or_benedictio_festivus}

\gscore{or_benedicamus_festis}

\end{document}