% !TEX TS-program = lualatex
% !TEX encoding = UTF-8

\documentclass[Session2024.tex]{subfiles}

\ifcsname preamble@file\endcsname
  \setcounter{page}{\getpagerefnumber{M-05_jeudi}}
\fi

\begin{document}
%% debut en page de droite
\addcontentsline{toc}{chapter}{Jeudi 25 juillet, saint Jacques}
\bigtitle{Saint Jacques, à la Messe}{jeudi 25 juillet}{Messe}

\gscore{in_mihi_autem_nimis_honorati}
\translation{\aa Selon moi, tes amis ont été plus qu’honorés, ô Dieu ; leur dignité de princes de l’Église a été puissamment établie.\\
\vv Seigneur, tu m’as éprouvé et tu m’as connu ; tu as connu mon entrée dans le repos et ma résurrection future.}

\smalltitle{Psalmodie de Tierce}
\gscore{an_dum_perambularet}
\translation{\aa Comme le Seigneur marchait
le long de la mer de Galilée,
il vit Jacques et Jean, et il les appela.}
\pagebreak
\smalltitle{Psaume 119}
\psalm{119}{1}
\smalltitle{Psaume 120}
\psalm{120}{1}
\needspace{3cm}
\smalltitle{Psaume 121}
\psalm{121}{1}

\vspace{2mm}
\gscore{ky_k4}
\gscore{ky_g4}
\gscore{gr_constitues_eos}
\translation{\rr Tu les établiras princes sur toute la terre ; ils se souviendront de ton nom, de génération en génération, Seigneur.\\
\vv À la place de tes pères, des fils te sont nés, c’est pourquoi les peuples te loueront.}
\gscore{al_ego_vos_elegi}
\translation{\vv Je vous ai établis afin que vous alliez, et que vous portiez du fruit, et que votre fruit demeure.}
\gscore{of_in_omnem_terram}
\translation{\rr Leur bruit s’est répandu dans toute la terre, et leurs paroles jusqu’aux extrémités du monde.}

~

\gscore{ky_s4}
\gscore{ky_a4}
\newpage
\gscore{co_ego_vos_elegi}
\translation{\aa Je vous ai établis afin que vous alliez, et que vous portiez du fruit, et que votre fruit demeure.\\
\vv L'amour du Seigneur, sans fin je le chante ; ta fidélité, je l'annonce d'âge en âge.\\
\vv «Avec mon élu, j'ai fait une alliance, j'ai juré à David, mon serviteur.»}

\bigtitle{Saint Jacques, à Sexte}{jeudi 25 juillet}{Sexte}

\smallscore{or_dia_ferialis}

\gscore{hy_rector_potens_festivus}
\translation{\colored{M}aître puissant, Dieu Vérité, tu règles la marche du temps, tu formes l’aube en sa clarté, au midi tu donnes ses flammes.\\
\colored{E}teins le feu des dissensions, calme la fièvre du péché, apporte à nos corps la santé, à nos cœurs, la paix véritable.\\
\colored{E}xauce-nous, Père très bon, et toi, le Fils égal au Père, avec l’Esprit Consolateur, régnant pour les siècles des siècles.}

\newpage

\gscore{an_sedere_autem}
\translation{\aa Siéger à ma droite et à ma gauche,
ce n’est pas à moi de l’accorder ;
il y a ceux pour qui cela est préparé par mon Père.}
\smalltitle{Psaume 122}
\psalm{122}{7}
\smalltitle{Psaume 123}
\psalm{123}{7}
\smalltitle{Psaume 124}
\psalm{124}{7} % TODO on peut renvoyer au mardi pour cette psalmodie

\newpage

\capitulum{Ac 5, 12a. 14}
{Per manus apostolórum fiébant signa et prodígia multa in plebe. †
Magis autem addebántur credéntes Dómino, * multitúdines virórum ac
mulíerum.}
{Par les mains des Apôtres, beaucoup de signes et de prodiges s’accomplissaient
dans le peuple. De plus en plus, des foules d’hommes et de
femmes, en devenant croyants, s’attachaient au Seigneur.}

\versiculus{Custodiébant testimónia Dei}{Et præcépta eius}{Ils ont gardé les volontés du Seigneur}{Les lois qu’il leur donna}

\rubric{Oraison des Vêpres, p. \pageref{0725V}.}

\gscore{or_benedicamus_hm_festivus}

\bigtitle{Saint Jacques, à None}{jeudi 25 juillet}{None}

\smallscore{or_dia_ferialis}

\gscore{hy_rerum_deus_festivus}
\translation{\colored{D}ieu fort, soutien de l’univers, qui es en toi sans changement, tu fixes dans leur succession les temps que le soleil mesure.\\
\colored{D}onne-nous un soir lumineux où la vie ne décline pas, où, pour fruit de la Sainte Mort, resplendira toujours la gloire.\\
\colored{E}xauce-nous, Père très bon, et toi, le Fils égal au Père, avec l’Esprit Consolateur, régnant pour les siècles des siècles.}

\gscore{an_occidit_herodes}
\translation{\aa Hérode fit tuer Jacques,
frère de Jean, par le glaive.}
\smalltitle{Psaume 125}
\psalm{125}{8}
\smalltitle{Psaume 126}
\psalm{126}{8}
\smalltitle{Psaume 127}
\psalm{127}{8}

\capitulum{Ac 5, 41-42}
{Ibant apóstoli gaudéntes a conspéctu concílii, † quóniam digni hábiti
sunt pro nómine contuméliam pati ; * et omni die in templo et circa
domos non cessábant docéntes et evangelizántes Christum, Iesum.}
{Les Apôtres, quittant le Conseil suprême, repartaient tout joyeux d’avoir
été jugés dignes de subir des humiliations pour le nom de Jésus. Tous
les jours, au Temple et dans leurs maisons, sans cesse, ils enseignaient
et annonçaient la Bonne Nouvelle : le Christ, c’est Jésus.}

\versiculus{Gaudéte et exsultáte, dicit Dóminus}{Quia nómina vestra scripta sunt in cælis}{Réjouissez-vous, exultez, dit le Seigneur}{Vos noms sont inscrits dans les cieux}

\rubric{Oraison des Vêpres, p. \pageref{0725V}.}

\gscore{or_benedicamus_hm_festivus}

\needspace{5cm}
\bigtitle{Saint Jacques, aux Vêpres}{jeudi 25 juillet}{Vêpres}

\smallscore{or_dia_festivus}

\gscore{hy_exsultet_caelum_laudibus}
\translation{\colored{Q}ue la louange exulte au ciel, que la joie réponde sur terre ! car c’est la gloire des Apôtres que nous célébrons aujourd’hui.\\
\colored{V}ous, les justes juges des hommes et les vraies lumières du monde, voici les voeux de notre coeur : écoutez nos voix suppliantes.\\
\colored{V}ous qui pouvez fermer le ciel et délier pour nous ses portes, nous vous prions : dites le mot qui nous délie de tout péché.\\
\colored{P}uisque santé et maladie obéissent à vos paroles, guérissez notre cœur malade, à notre âme rendez vigueur.\\
\colored{A}insi, quand reviendra le Christ pour juger, à la fin des temps, il nous fera participer au bonheur qui n’a pas de fin.}

\needspace{3cm}
\smalltitle{Psaume 109}
\gscore{an_isti_sunt_viri_sancti}
\translation{\aa Ce sont eux les hommes saints
que le Seigneur a choisis dans une charité non feinte,
et il leur donna une gloire éternelle;
par leur doctrine, l’Eglise resplendit comme la lune par le soleil.}
\psalm{109}{7}

\smalltitle{Psaume 112}
\gscore{an_vos_estis}
\translation{\aa C'est vous qui êtes demeurés avec moi dans mes épreuves ;
et moi, je suis au milieu de vous comme celui qui sert.}
\psalm{112}{2}

\smalltitle{Psaume 115}
\gscore{an_isti_viventes_in_carne}
\translation{\aa Voici ceux qui dans leur vie terrestre ont planté l'Eglise par leur sang; leurs corps n'ont pas été enlevés de la terre, eux dont les mérites sont dans les cieux comme les âmes des saints.}
\psalm{115}{7}

\smalltitle{Psaume 125}
\gscore{an_iam_non_dicam_vos_servos}
\translation{\aa Je ne vous appelle plus serviteurs mais mes amis
car tout ce que j'ai appris de mon Père,
je vous l'ai fait connaître.}
\psalm{125}{1d}

\capitulum{Ep 4, 11-13}
{Christus dedit quosdam quidem apóstolos, quosdam autem
prophétas, † álios vero evangelístas, álios autem pastóres et doctóres *
ad instructiónem sanctórum in opus ministérii, in ædificatiónem córporis
Christi, † donec occurrámus omnes in unitátem fídei et agnitiónis Fílii
Dei, * in virum perféctum, in mensúram ætátis plenitúdinis Christi.}
{Les dons que le Christ a faits, ce sont les Apôtres, et aussi les prophètes,
les évangélisateurs, les pasteurs et ceux qui enseignent. De cette
manière, les fidèles sont organisés pour que les tâches du ministère
soient accomplies et que se construise le corps du Christ, jusqu’à ce
que nous parvenions tous ensemble à l’unité dans la foi et la pleine
connaissance du Fils de Dieu, à l’état de l’Homme parfait, à la stature
du Christ dans sa plénitude.}

\gscore{rb_annuntiate_inter_gentes}
\translation{\rr Racontez à tous les peuples la gloire du Seigneur.
\vv À toutes les nations ses merveilles.}

\newpage
\smalltitle{Magnificat}
\gscore{an_quicumque_voluerit_inter_vos}
\translation{\aa Celui qui veut devenir grand parmi vous sera votre serviteur ;
et celui qui veut être parmi vous le premier sera votre esclave.}
\incipit{magn}{8gsol}
\psalm{magn}{8sol}
\gscore{an_quicumque_voluerit_inter_vos}

\smalltitle{Intercession}
\smallscore{or_kyrie_festivus}
\smallscore{or_pater_festivus}

\label{0725V}
\oratio{
Omnípotens sempitérne Deus, qui Apostolórum tuórum primítias
beáti Iacóbi sánguine dedicásti, † da, quǽsumus, Ecclésiæ tuæ ipsíus
confessióne firmári, * et iúgiter patrocíniis confovéri. Per Dóminum.
}{
Dieu éternel et tout-puissant, tu as consacré l’offrande du bienheureux
Jacques, le premier de tes Apôtres à verser pour toi son sang, accorde à
ton Église de trouver dans son témoignage une force, et dans sa protection
un appui constant.
}

\blessing

\gscore{or_benedicamus_festis}

\end{document}